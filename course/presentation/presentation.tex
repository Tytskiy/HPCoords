\documentclass[fleqn, xcolor=x11names]{beamer}
\usetheme{myamsterdam} %тема\
\usecolortheme{default}
\usepackage[utf8]{inputenc}
\usepackage[russian]{babel}
\usepackage[OT1]{fontenc}
\usepackage{amsmath}
\usepackage{amsfonts}
%\usepackage{amssymb}
\usepackage{hyperref}
\usepackage{graphics, graphicx}
\usepackage{color}
\usepackage{enumerate}

\usepackage{epigraph}
\usepackage{makecell}
% \usepackage{citehack} 
\usepackage{pgfplots}
\usepackage{tikz}
\usetikzlibrary{patterns}

\usepackage{minted}
\usemintedstyle{default}
%my package
\graphicspath{{../figures/}}

\beamertemplatenavigationsymbolsempty

\setbeamertemplate{itemize item}[ball]
\setbeamertemplate{itemize subitem}[ball]
\definecolor{my_blue}{RGB}{0, 0, 100}

\setbeamercolor{palette tertiary}{use=structure,fg=white,bg=my_blue}
\setbeamercolor{block title}{bg=my_blue,fg=white}

\setbeamercolor{item projected}{fg=white,bg=my_blue}
\setbeamercolor{subitemitem projected}{fg=white,bg=my_blue}
%\usefonttheme[onlylarge]{structurebold} % названия и текст в колонтитулах выводится полужирным шрифтом.
\usefonttheme[onlymath]{serif}  % привычный шрифт для математических формул
%\setbeamerfont*{frametitle}{size=\normalsize,series=\bfseries} % шрифт заголовков слайдов
\usepackage[nopar]{lipsum} %для генерации большого текста

\newminted[pcode]{python}{baselinestretch=1, fontsize=\small}
\newmintinline[pinline]{python3}{baselinestretch=1}
%\definecolor{bg}{rgb}{0.95,0.95,0.95}
%\newminted[lcode]{latex}{baselinestretch=1, bgcolor=bg}
\newmintinline[linline]{latex}{baselinestretch=1}

\usepackage{tcolorbox}
\tcbuselibrary{minted,skins}

\newtcblisting{lcode}{
    listing engine=minted, %use minted for highlight
    colback=lcodebg, %background color
    colframe=black!50, %width of frame
    listing only,
    minted style=colorful,
    minted language=latex,
    minted options={linenos=false,texcl=true}, %lines - number of lines
    left=1mm,
}
\definecolor{lcodebg}{rgb}{0.95,0.95,0.95}

\usepackage{tikz}
\usetikzlibrary{arrows,positioning}
\usepackage{listings}
\lstset{language=Python}

\newcommand{\real}{\mathbb{R}}
\newcommand{\norm}{\mathop{\rm norm}\limits}
\newcommand{\softmax}{\mathop{\rm softmax}\limits}

\definecolor{beamer@blendedblue}{rgb}{0.037,0.366,0.75}

\title{\bfseries График в параллельных осях}
\author[Тыцкий В.И.]{Студент: Тыцкий В.И.\\[1ex]  {\small Научный руководитель: Майсурадзе А.И.}}
\institute[ВМК МГУ]{МГУ имени М. В. Ломоносова, факультет ВМК, кафедра ММП}
\date{}

\begin{document}

\begin{frame}
    \titlepage
\end{frame}

 \begin{frame}{Оглавление}
     \tableofcontents
\end{frame}

\section{Введение}

\begin{frame}{Примеры диаграмм}
    \centering
    \begin{tabular}{cc}
        \includegraphics[width=4cm]{example_1.png} &
        \includegraphics[width=4cm]{example_2.png}   \\
        \includegraphics[width=4cm]{example_3.png} &
        \includegraphics[width=4cm]{example_4.png}
    \end{tabular}
\end{frame}

\begin{frame}{Историческая справка}
    \begin{quote}
        The value of data visualization is not seeing “zillions” 
    of objects but rather recognizing relations among them.
    \rightline{Alfred Inselberg}
    \end{quote}
    \vspace{30px}
    \begin{itemize}
        \item Параллельные координаты были известны еще в 19-ом веке
        \item В 1980-ых были популяризированы Альфредом Инсельбергом
    \end{itemize}
\end{frame}

\begin{frame}{Классический график в параллельных осях}
    \begin{figure}[htb]
        \centering
        \includegraphics[width=10.5cm]{classic_pc.pdf}
    \end{figure}

\end{frame}

\section{Модификации}

\begin{frame}{Модификации: кластеры}

    \vspace{10px}

    Кластер — класс родственных элементов статистической совокупности.

    \begin{figure}[htb]
        \centering
        \includegraphics[width=10.5cm]{color_pc.pdf}
    \end{figure}

    Чаще всего именно в таком виде используют график в параллельных осях.
\end{frame}

\begin{frame}{Cглаживание линий}

    \vspace{10px}

    Человеку проще воспринимать гладкие линии, поэтому читаемость графика заметно возрастает.

    \begin{figure}[htb]
        \centering
        \includegraphics[width=10.5cm]{smooth_pc.pdf}
    \end{figure}
\end{frame}

\begin{frame}{Cвязывание линий}
    \begin{figure}[htb]
        \centering
        \includegraphics[width=10.5cm]{bundle_0.3_pc.pdf}
    \end{figure}
\end{frame}

\begin{frame}{Cвязывание линий}  
    \begin{figure}[htb]
        \centering
        \includegraphics[width=10.5cm]{bundle_0.01_pc.pdf}
    \end{figure}
\end{frame}

\begin{frame}{Иерархические графики}

    Изображаем статистики распределений соответствующих кластеров (std, min, max, mean)
    вместо отрисовки каждого объекта.

    \begin{figure}[htb]
        \centering
        \includegraphics[width=5cm]{hierarchical_1.png}
    \end{figure}
\end{frame}

\begin{frame}{Иерархические графики}
    Пусть $X=(x_1,\ldots,x_n)$ -- выборка, где $x_i\in \mathbb{R}^n$. 

    \vspace{10px}

    Назовем множество $P$ m-разбиением множества X на m-подмножеств $\{P_1,\ldots,P_m\}$
    такое, что:
    \begin{align}
        \notag &1. P_i \cap  P_j = \oslash, \hspace{10px} \forall i,j =\overline{1,m} \\
        \notag &2. \bigcup\limits_{i=1}^{m} P_i = X 
    \end{align}

    Организуем иерархическую структуру в виде дерева, где корнем является $X$, 
    а каждая вершина сопоставлена элементу разбиения родительской вершины.
\end{frame}

\begin{frame}{Пример иерархического разбиения}
    \begin{figure}[htb]
        \centering
        \includegraphics[width=10cm]{hierarchical_graph.png}
    \end{figure}

\end{frame}


\begin{frame}{Иерархические графики}

    Регулируя глубину, мы добавляем/уменьшаем количество кластеров на графике

    \begin{figure}[htb]
        \centering
        \includegraphics[width=10.5cm]{hierarchical_2.png}
    \end{figure}
\end{frame}

\begin{frame}{3D}
    \begin{tabular}{cc}
        \centering
        \includegraphics[width=6cm]{3d_pc.png} &
        \includegraphics[width=3.5cm]{multi_relational_pc.png}   \\
        3D parallel coordinates & \makecell{3D multi-relational \\ parallel coordinates}
    \end{tabular}
\end{frame}

\section{Проблемы построения}

\begin{frame}{Естественные вопросы при построении}
    \begin{itemize}
        \item В каком порядке расположить оси?
        \item В какую сторону направлять оси?
        \item Как много объектов отобразить?
        \item Какой масштаб выбрать для каждой оси?
    \end{itemize}
\end{frame}

\begin{frame}{Выбор направлений и порядка осей}
    \begin{figure}[htb]
        \centering
        \includegraphics[width=10.5cm]{upgrade_1.pdf}
    \end{figure}
\end{frame}

\begin{frame}{Выбор направлений и порядка осей}
    \begin{figure}[htb]
        \centering
        \includegraphics[width=10.5cm]{upgrade_2.pdf}
    \end{figure}
\end{frame}

\begin{frame}{Выбор направлений и порядка осей}
    \begin{figure}[htb]
        \centering
        \includegraphics[width=10.5cm]{upgrade_3.pdf}
    \end{figure}
\end{frame}


\begin{frame}{Выбор направлений и порядка осей}
    \begin{figure}[htb]
        \centering
        \includegraphics[width=10.5cm]{upgrade_4.pdf}
    \end{figure}
\end{frame}

\begin{frame}{Влияение количества объектов на читаемость}
    \begin{figure}[htb]
        \centering
        \includegraphics[width=10.5cm]{base_good_clustering.pdf}
    \end{figure}
\end{frame}

\begin{frame}{Влияение количества объектов на читаемость}
    \begin{figure}[htb]
        \centering
        \includegraphics[width=10.5cm]{good_clustering.pdf}
    \end{figure}
\end{frame}

\begin{frame}{Влияение количества объектов на читаемость}
    \begin{figure}[htb]
        \centering
        \includegraphics[width=10.5cm]{base_bad_clustering.pdf}
    \end{figure}
\end{frame}

\begin{frame}{Влияение количества объектов на читаемость}
    \begin{figure}[htb]
        \centering
        \includegraphics[width=10.5cm]{bad_clustering.pdf}
    \end{figure}
\end{frame}

\begin{frame}{Резюмируя}
    \begin{itemize}
        \item Изменение степени прозрачности линий.
        \item Использование гладких линий.
        \item Связывание линий в рамках кластеров.
        \item Отображение лишь части объектов.
        \item \textbf{Изменение порядка и направления осей.}
    \end{itemize}
\end{frame}

\section{Методы выбора порядка}

\begin{frame}{Корреляция}
    Пусть даны две выборки $X=(x_1, \ldots ,x_n), Y=(y_1,\ldots ,y_n)$.

    \begin{block}{Корреляция Пирсона}
            \centering
            $\rho_{XY} = 
            \frac{\sum\limits_{i=1}^{n}(x_i-\overline x)(y_i-\overline y)}
            {\sqrt{\sum\limits_{i=1}^{n}(x_i-\overline x)^2\sum\limits_{i=1}^{n}(y_i-\overline y)^2}}, 
            \hspace{15px} \left\lvert \rho_{XY} \right\rvert \leq 1$
    \end{block}

    Пусть $R_i$ -- ранг наблюдения $x_i$, $S_i$ -- ранг наблюдения $y_i$

    \begin{block}{Корреляция Спирмена}
        \centering
        $r_{XY} = 
        \frac{\sum\limits_{i=1}^{n}(R_i-\overline R)(S_i-\overline S)}
        {\sqrt{\sum\limits_{i=1}^{n}(R_i-\overline R)^2\sum\limits_{i=1}^{n}(S_i-\overline S)^2}},
        \hspace{15px} \left\lvert r_{XY} \right\rvert \leq 1$
    \end{block}
    $\rho_{XY} = 0$, $r_{XY} = 1$, где $X=Y^2$ и $X$ симметрично распределена относительно нуля.

\end{frame}

\begin{frame}{}

    \begin{block}{Задача о самом длинном пути}
    Это задача поиска простого пути максимальной длины в заданном графе.
    Является NP-трудной и не может быть решена за 
    полиномиальное время для произвольных графов.
    \end{block}

    \vspace{10px}

    Пусть $X=(x_1,\ldots,x_n)$ -- выборка, где $x_i\in \mathbb{R}^n$. 

    \vspace{10px}

    Построим связный граф $G(V,E)$, где каждая вершина $u^i \in V$ соответствует
    i-ой координатe (i-ой оси на графике), а каждому ребру $\{u^i,u^j\} \in E$ сопоставим вес равный $|r_{x^i,x^j}|$.

\end{frame}

\begin{frame}{}

    \begin{itemize}
        \item Простейшим перебором задача решается за $O(n!)$
        \item Можно свести к задаче коммивояжера.
        \item С помощью методов динамического программирования можно улучшить асимптотику решения.
    \end{itemize}

    \vspace{10px}

    \begin{tabular}{cc}
        \centering
        \includegraphics[width=5cm]{graph_example_1.png} &
        \includegraphics[width=5cm]{graph_example_2.png}   \\
    \end{tabular}
\end{frame}

\section{О библиотекe}

\begin{frame}{Обзор текущих средств}

    \begin{itemize}
        \item На Python есть простейшая реализация лишь в библиотеке pandas!
        \item ELKI, GGobi, Mondrian, Orange и ROOT.
        \item Parcoords.js интерактивная библиотека на JavaScript.
    \end{itemize}

\end{frame}

\begin{frame}{Цели}
    \begin{itemize}
        \item Дать возможность исследователям ''безболезненно'' использовать график
              в параллелльных осях.
        \item Построение красивых и информативных графиков из ''коробки''.
        \item Реализация всевозможных видов данных графиков.
    \end{itemize}
\end{frame}

\begin{frame}{Технические подробности}
    \begin{itemize}
        \item Статические графики.
        \item Библиотека пишется на языке Python на базе matplotilb.
        \item Простой высокоуровневый интерфейс. Как и в библиотеке seaborn методы могут принимать pandas.DataFrame,
              обычные numpy массивы или списки -- для всего единый интерфейс.
    \end{itemize}
\end{frame}

\begin{frame}{Возможности}
    \begin{itemize}
        \item Построение классических графиков в параллельных осях
              \begin{itemize}
                  \item Возможность рисовать гладкие линии.
                  \item Возможность ''связывания'' линий кластеров.
                  \item Возможность ''связывания'' линий на основе близости.
              \end{itemize}
              \vspace{10px}
        \item Построение иерархических графиков
              \begin{itemize}
                  \item Отрисовка полупрозрачного градиента.
                  \item Работа с иерархическими кластерами.
                  \item Изображение распределения с помощью градиента.
              \end{itemize}
    \end{itemize}
\end{frame}

\begin{frame}{Дополнительные возможности}
    \begin{itemize}
        \item выделение подмножества линий в  диапазоне значений одной из осей.
        \item нахождение оптимального расположения осей.
        \item создание иерархических кластеров на основе входящей выборки.
    \end{itemize}
\end{frame}

\begin{frame}{Итоги (после первого семестра)}
    \begin{itemize}
        \item Возможность рисовать гладкие линии. \textbf{Пока что не добавлен параметр задающий вид кривой}.
        \item Возможность ''связывания'' линий кластеров.
              Добавлен непрерывный параметр задающий степень связывания.
        \item \textbf{Возможность связывания линий на основе близости не реализована}
        \item Интерфейс для пользователя практически полностью повторяет реализацию seaborn.\footnote{
                  Большинство графиков в презентации нарисованы с помощью данной библиотеки.}
    \end{itemize}
\end{frame}
\end{document}